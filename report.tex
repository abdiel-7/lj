\documentclass[11pt]{article}
\usepackage{amsfonts}
\usepackage{amscd}
\usepackage{amssymb}
\usepackage{amsthm}
\usepackage{amsmath, xspace}
\usepackage{blkarray}
\usepackage{bm}
\usepackage{cancel}
\usepackage{amsmath}
\usepackage{color}
\usepackage{graphics}
\usepackage{graphicx}
\usepackage{enumitem}
\usepackage{fancyhdr}
\pagestyle{fancy}
\usepackage{mathdots}
\usepackage{mathrsfs}
\usepackage{multicol}
\usepackage{stmaryrd}
%\usepackage{youngtab}
\usepackage{ytableau}
%\usepackage{refcheck}
\usepackage[all]{xy}
%\usepackage[neveradjust]{paralist}
\usepackage{amsthm, amssymb, graphicx}
\usepackage[plainpages]{hyperref}
\usepackage{varwidth}
% \usepackage{subcaption}


\usepackage{lscape}
\usepackage{listings}
\usepackage{xcolor}
\usepackage[capitalise]{cleveref}
\lstset{
language=Python, 
% numbers=left, 
numberstyle=\tiny\color{gray}, 
keywordstyle=\color{blue!70}, 
commentstyle=\color{green!50!black}, 
stringstyle=\color{red!60!brown}, 
backgroundcolor=\color{gray!10}, 
frame=single,
breaklines=true, 
tabsize=4
}

\theoremstyle{plain}
\newtheorem{thm}{Theorem}[section]
\newtheorem{lemma}[thm]{Lemma}
\newtheorem{prop}[thm]{Proposition}
\newtheorem{cor}[thm]{Corollary}
\theoremstyle{Definition}
\newtheorem{defn}{Definition}[section]
\newtheorem{remark}[thm]{Remark}
\theoremstyle{example}
\newtheorem{example}{Example}[section]
\newtheorem*{question}{Question}
\newtheorem{conj}[thm]{Conjecture}
\theoremstyle{remark}
\newtheorem*{ack}{Acknowledgments}
\numberwithin{equation}{section}

\setlength{\evensidemargin}{1in} 
\addtolength{\evensidemargin}{-1in}
\setlength{\oddsidemargin}{1in} 
\addtolength{\oddsidemargin}{-1in} 
\setlength{\topmargin}{1in} 
\addtolength{\topmargin}{-1.5in}

\setlength{\textwidth}{17cm} \setlength{\textheight}{23cm}
\setlength{\headwidth}{14cm}
\setlength{\headheight}{13.6pt}

\providecommand{\keywords}[1]{\textbf{\textit{Key words---}} #1}
\newcommand{\btimes}{\mathbin{\rotatebox[origin=c]{90}{$\ltimes$}}}
\newcommand{\utimes}{\mathbin{\rotatebox[origin=c]{-90}{$\ltimes$}}}

\def\cA{\mathcal{A}}
\def\cB{\mathcal{B}}
\def\cC{\mathcal{C}}
\def\cD{\mathcal{D}}
\def\cE{\mathcal{E}}
\def\cF{\mathcal{F}}
\def\cG{\mathcal{G}}
\def\cH{\mathcal{H}}
\def\cI{\mathcal{I}}
\def\cJ{\mathcal{J}}
\def\cK{\mathcal{K}}
\def\cL{\mathcal{L}}
\def\cM{\mathcal{M}}
\def\cN{\mathcal{N}}
\def\cO{\mathcal{O}}
\def\cP{\mathcal{P}}
\def\cQ{\mathcal{Q}}
\def\cR{\mathcal{R}}
\def\cS{\mathcal{S}}
\def\cT{\mathcal{T}}
\def\cU{\mathcal{U}}
\def\cV{\mathcal{V}}
\def\cW{\mathcal{W}}
\def\cX{\mathcal{X}}
\def\cY{\mathcal{Y}}
\def\cZ{\mathcal{Z}}

\def\AA{\mathbb{A}}
\def\BB{\mathbb{B}}
\def\CC{\mathbb{C}}
\def\DD{\mathbb{D}}
\def\EE{\mathbb{E}}
\def\FF{\mathbb{F}}
\def\GG{\mathbb{G}}
\def\HH{\mathbb{H}}
\def\II{\mathbb{I}}
\def\JJ{\mathbb{J}}
\def\KK{\mathbb{K}}
\def\LL{\mathbb{L}}
\def\MM{\mathbb{M}}
\def\NN{\mathbb{N}}
\def\OO{\mathbb{O}}
\def\PP{\mathbb{P}}
\def\QQ{\mathbb{Q}}
\def\RR{\mathbb{R}}
\def\SS{\mathbb{S}}
\def\TT{\mathbb{T}}
\def\UU{\mathbb{U}}
\def\VV{\mathbb{V}}
\def\WW{\mathbb{W}}
\def\XX{\mathbb{X}}
\def\YY{\mathbb{Y}}
\def\ZZ{\mathbb{Z}}



\def\fa{\mathfrak{a}}
\def\fb{\mathfrak{b}}
\def\fc{\mathfrak{c}}
\def\fd{\mathfrak{d}}
\def\fe{\mathfrak{e}}
\def\ff{\mathfrak{f}}
\def\fg{\mathfrak{g}}
\def\fh{\mathfrak{h}}
\def\fj{\mathfrak{j}}
\def\fk{\mathfrak{k}}
\def\fl{\mathfrak{l}}
\def\fm{\mathfrak{m}}
\def\fn{\mathfrak{n}}
\def\fo{\mathfrak{o}}
\def\fp{\mathfrak{p}}
\def\fq{\mathfrak{q}}
\def\fr{\mathfrak{r}}
\def\fs{\mathfrak{s}}
\def\ft{\mathfrak{t}}
\def\fu{\mathfrak{u}}
\def\fv{\mathfrak{v}}
\def\fw{\mathfrak{w}}
\def\fx{\mathfrak{x}}
\def\fy{\mathfrak{y}}
\def\fz{\mathfrak{z}}


\def\fU{\mathfrak{U}}

\def\fgl{\mathfrak{gl}}
\def\fsl{\mathfrak{sl}}

\def\ad{\mathrm{ad}}
\def\Aut{\mathrm{Aut}}
\def\Card{\mathrm{Card}}
\def\diag{\mathrm{diag}}
\def\dim{\mathrm{dim}}
\def\End{\mathrm{End}}
\def\ev{\mathrm{ev}}
\def\Hom{\mathrm{Hom}}
\def\GL{\mathrm{GL}}
\def\id{\mathrm{id}}
\def\Ind{\mathrm{Ind}}
\def\ind{\mathrm{ind}}
\def\Inf{\mathrm{Inf}}
\def\Irr{\mathrm{Irr}}
\def\Rad{\mathrm{Rad}}
\def\Res{\mathrm{Res}}
\def\Resf{\mathrm{Resf}}
\def\tr{\mathrm{tr}}
\def\height{\mathrm{ht}}
\def\Tr{\mathrm{Tr}}
\def\wt{\mathrm{wt}}


%%%%% Diagrams %%%%% 
\usepackage{etex} %fixes the fight that pictex has with every other drawing package
\usepackage{pictexwd}
%\input xy
%\xyoption{all}
\usepackage{tikz}
	\usepgflibrary[patterns] % ConTEXt and pure pgf 
	\usetikzlibrary{patterns} % LATEX and plain TEX when using TikZ 
	\usepgflibrary{shapes.geometric}
        \usetikzlibrary{arrows, calc, positioning}

\newcommand{\TikZ}[1]{
\begin{matrix}\begin{tikzpicture}#1\end{tikzpicture}\end{matrix}
}

\def\ShiftX{\pgftransformxshift}
\def\ShiftY{\pgftransformyshift}

%PARTITIONS
\newcounter{r}
\newcounter{s}




%Shortcut for a partition, to be used in TikZ environment.
%Example: \begin{tikzpicture} \Part{5,4,2} \end{tikzpicture}
\newcommand\Part[1]{
        \setcounter{r}{1}
	 \foreach \x in {#1}{
 	{\ifnum\value{r}=1
		\draw (0,\value{r}-1)--(\x,\value{r}-1); 
		\fi}
	\draw (0,\value{r}) to (\x,\value{r});
   	\foreach \y in {0, ..., \x} {\draw (\y,\value{r})--(\y,\value{r}-1);}
	\addtocounter{r}{1}
 }}
 \def\PartUNIT{.175}
%Self-contained tikz images for \Part above, to be used in math mode.
%Example: $\PART{5,4,2}$ or $\sPART{5,4,2}$
\newcommand{\PART}[1]{%Good for displayed, moderate sized partitions.
\begin{matrix}
\begin{tikzpicture}[scale=.35, yscale=-1] 
	\Part{#1}\node at (.5,0){};
\end{tikzpicture}
\end{matrix}
}
\newcommand{\sPART}[1]{%Good for subscripts.
\begin{matrix}
\begin{tikzpicture}[scale=.175, yscale=-1] 
	\Part{#1}\node at (.5,0){};
\end{tikzpicture}
\end{matrix}
}

%Outline of a partition. Use in tikzpicture environment. 
%Example: \begin{tikzpicture} \EmptyPart{5,4,2} \end{tikzpicture}
\newcommand\EmptyPart[1]{
        \setcounter{r}{0}
        \setcounter{s}{0}
	 \foreach \x in {#1}{
	 \draw (\value{s}, \value{r}) to (\x, \value{r}) to (\x, \value{r}+1);
%   	\foreach \y in {0, ..., \x} {\draw (\y,\value{r})--(\y,\value{r}-1);}
	\addtocounter{r}{1}
        \setcounter{s}{\x}
 }
 \draw (\value{s}, \value{r}) to (0, \value{r}) to (0,0);
 }
 %Stand-alone code for \EmptyPart. Use in math environment.
 %Example: $\ePART{5,4,2}$
 \newcommand{\ePART}[1]{
\begin{matrix}
\begin{tikzpicture}[scale=.35, yscale=-1] 
	\EmptyPart{#1}\node at (.5,0){};
\end{tikzpicture}
\end{matrix}
}
 
%Shortcut for a tableau, to be used in TikZ environment.
%Example: \begin{tikzpicture} \Tableau{{1,1,1,2,3},{2,2,4,4},{3,4}} \end{tikzpicture}
\newcommand\Tableau[1]{
        \foreach \x [count = \c from 1] in {#1} {
		\foreach \y [count = \d from 1] in \x{
			\node at (\d-.5,\c-.5) {\scriptsize$\y$}; %Want to change the color of the numbers? insert \color{???} just after \scriptsize
			\draw (\d,\c) to (\d,\c-1);
			{\ifnum\d=1
				\draw (0,\c) to (0,\c-1);
				\fi}
			\setcounter{r}{\d}
		}
		{\ifnum\c=1
			\draw (0,0)--(\value{r},0);
			\fi}
		\draw(0,\c) to (\value{r},\c);
		\setcounter{s}{\c}}}
		
 %Stand-alone code for \Tableau. Use in math environment.
 %Best for displayed environments.
 %Example: $\TBL{{1,1,1,2,3},{2,2,4,4},{3,4}}$
\newcommand{\TBL}[1]{
\begin{matrix}
\begin{tikzpicture}[scale=.35, yscale=-1] 
	\Tableau{#1}\node at (.5,0){};
\end{tikzpicture}
\end{matrix}
}
%Shortcut for a small tableau, to be used in TikZ environment.
% Best for in-line or subscripts.
%Example: \begin{tikzpicture} \sTableau{{1,1,1,2,3},{2,2,4,4},{3,4}} \end{tikzpicture}
\newcommand\sTableau[1]{
        \foreach \x [count = \c from 1] in {#1} {
		\foreach \y [count = \d from 1] in \x{
			\node at (\d-.5,\c-.5) {\tiny$\y$}; %Want to change the color of the numbers? insert \color{???} just after \scriptsize
			\draw (\d,\c) to (\d,\c-1);
			{\ifnum\d=1
				\draw (0,\c) to (0,\c-1);
				\fi}
			\setcounter{r}{\d}
		}
		{\ifnum\c=1
			\draw (0,0)--(\value{r},0);
			\fi}
		\draw(0,\c) to (\value{r},\c);
		\setcounter{s}{\c}}}
%Stand-alone code for \sTableau. Use in math environment.
%Best for in-line or subscripts.
%Example: $\sTBL{{1,1,1,2,3},{2,2,4,4},{3,4}}$		
\newcommand{\sTBL}[1]{
\begin{matrix}
\begin{tikzpicture}[scale=.25, yscale=-1] 
	\sTableau{#1}\node at (.5,0){};
\end{tikzpicture}
\end{matrix}
}



\newcommand{\PartB}[1]{
 \foreach \x [count=\s from 1] in {#1}{
 	{\ifnum\s=1
		\draw (0,\s-1)--(\x,\s-1); 
		\fi}
   \draw (0,\s) to (\x,\s);
   \foreach \y in {0, ..., \x} {\draw (\y,\s)--(\y,\s-1);}
 }}

\def\UNITB{.25} %Sets the scale of your partitions 
%Whenever there's "yscale = - ...", that turns  the coordinates upside-down, to agree with how we count rows of partitions. 

%Self-contained tikz images for \Part and \fPart above. 
\newcommand{\PARTB}[1]{
\begin{tikzpicture}[xscale=\UNITB, yscale=-\UNITB] 
	\Part{#1}
\end{tikzpicture}
}
\newcommand{\fPARTB}[2]{
\begin{tikzpicture}[xscale=\UNITB, yscale=-\UNITB]
	\fPart{#1}{#2}
\end{tikzpicture}
}
%This is the lattice behind all the examples below. 
\def\Lattice{
%Sets the coordinates for the partitions of 0 to 5, grouped by level. Adjust y-coordinates to spread out levels piecemeal, or just adjust overall yscale in the TikZ environment.
	\coordinate (0) at (0,.4); %level 0
	\coordinate (11) at (0,1.2); %level 1
	\coordinate (21) at (-1,2);\coordinate (22) at (1,2); %level 2
	\foreach \x in {1, ..., 3}{\coordinate (3\x) at (-4+2*\x,3);} %level 3
	\foreach \x in {1, ..., 5}{\coordinate (4\x) at (-6+2*\x,4.5);} %level 4
	\foreach \x in {1, ..., 7}{\coordinate (5\x) at (-8+2*\x,6);} %level 5
%edges in lattice:
	%level 0 -> 1
	\draw (0) to node[left] {$a$} (11); 
	%level 1 -> 2
	\draw (11) to node[above left] {$b$} (21) 
		(11) to node[above right] {$c$} (22);
	\draw (21) to node[above left] {$e$} (31) 
	%level 2 -> 3
		(21) to node[below left] {$f$} (32) 
		(22) to node[above left] {$g$} (32) 
		(22) to node[above right] {$h$} (33);
	%level 3 -> 4
	\draw (31) to node[above left] {$i$}  (41) 
		(31) to node[left] {$j$}  (42) 
		(32) to node[above left] {$k$}  (42) 
		(32) to node[left] {$l$}  (43) 
		(32) to node[above right] {$m$}  (44) 
		(33) to node[right] {$n$}  (44) 
		(33) to node[above right] {$o$}  (45);
	%level 4 -> 5
	\draw (41) to node[left] {$p$}  (51) 
		(41) to node[left] {$q$}  (52) 
		(42) to node[left] {$r$}  (52) 
		(42) to node[left] {$s$}  (53) 
		(42) to[bend left=5] node[pos=.35, above right] {$t$}  (54) 
		(43) to[bend right=5] node[pos=.35, below right] {$u$}  (53)
		(43) to[bend left=5] node[pos=.35, above right] {$v$}  (55) 
		(44) to[bend right=5] node[pos=.35, below right] {$w$}  (54) 
		(44) to[bend right=5] node[right] {$x$}  (55) 
		(44) to node[above right] {$y$}  (56) 
		(45) to node[right] {$z$}  (56) 
		(45) to node[above right] {$?$}  (57);
%partitions in lattice:
\begin{scope}[every node/.style={fill=white}]
	%level 0
	\node at (0) {$\emptyset$};
	%level 1
	\node at (11) {\PARTB{1}};
	%level 2
	\node at (22) {\PARTB{1,1}};
	\node at (21) {\PARTB{2}};
	%level 3
	\node at (33) {\PARTB{1,1,1}};
	\node at (32) {\PARTB{2,1}};
	\node at (31) {\PARTB{3}};
	%level 4
	\node at (45) {\PARTB{1,1,1,1}};
	\node at (44) {\PARTB{2,1,1}};
	\node at (43) {\PARTB{2,2}};
	\node at (42) {\PARTB{3,1}};
	\node at (41) {\PARTB{4}};
	%level 5
	\node at (57) {\PARTB{1,1,1,1,1}};
	\node at (56) {\PARTB{2,1,1,1}};
	\node at (55) {\PARTB{2,2,1}};
	\node at (54) {\PARTB{3,1,1}};
	\node at (53) {\PARTB{3,2}};
	\node at (52) {\PARTB{4,1}};
	\node at (51) {\PARTB{5}};
\end{scope}
\foreach \x in {-4,0,4} {\node at (\x , 6.75) {$\vdots$}; }
}

%%%%%% Diagrams %%%%% 
%\usepackage{etex} %fixes the fight that pictex has with every other drawing package
%\usepackage{pictexwd}
%\usepackage{tikz}
%\usepgflibrary{shapes.geometric}
%\usetikzlibrary{arrows, calc, positioning}
%%\usetikzlibrary{arrows}
%\def\ShiftX{\pgftransformxshift}
%\def\ShiftY{\pgftransformyshift}

\tikzstyle{V}=[draw, fill =black, circle, inner sep=0pt, minimum size=1.5pt]
\tikzstyle{wV}=[draw, fill =white, circle, inner sep=0pt, minimum size=4.5pt]
\tikzstyle{bV}=[draw, fill =black, circle, inner sep=0pt, minimum size=4.5pt]
\tikzstyle{over}=[draw=white,double=black,line width=2pt, double distance=.5pt]


%BRAIDS:
\def\Over[#1,#2][#3,#4]{ %1,2=start position; 3,4=end position
	\draw[style=over]   (#2,#1) .. controls ++(#4*.5-#2*.5,0) and ++(-#4*.5+#2*.5,0) .. (#4,#3);}
\def\Under[#1,#2][#3,#4]{ %1,2=start position; 3,4=end position
	\draw  (#2,#1) .. controls ++(#4*.5-#2*.5,0) and ++(-#4*.5+#2*.5,0) .. (#4,#3);}
\def\Cross[#1,#2][#3,#4]{%Mimic over, under follows
	\Under[#3,#2][#1,#4]\Over[#1,#2][#3,#4]}
%\def\Over[#1,#2][#3,#4]{ %1,2=start position; 3,4=end position
%	\draw[style=over]   (#1,#2) .. controls ++(0,#4*.5-#2*.5) and ++(0,-#4*.5+#2*.5) .. (#3,#4);}
%\def\Under[#1,#2][#3,#4]{ %1,2=start position; 3,4=end position
%	\draw  (#1,#2) .. controls ++(0,#4*.5-#2*.5) and ++(0,-#4*.5+#2*.5) .. (#3,#4);}
%\def\Cross[#1,#2][#3,#4]{%Mimic over, under follows
%	\Under[#3,#2][#1,#4]\Over[#1,#2][#3,#4]}


%\def\Ez[#1]{\draw [over, bend left=75] (1,#1+1) to (1-.4,#1+.7)  (1-.4,#1+.3)  to (1,#1) ;
%		\draw[densely dotted]  (1-.4,#1)--(1-.4,#1+1) ; \node[V] at (1-.4,#1+.3){}; \node[V] at  (1-.4,#1+.7){};}
%\def\Ek[#1][#2]{\draw [over, bend right=75] (#2,#1+1) to (#2+.4,#1+.7)  (#2+.4,#1+.3)  to (#2,#1) ;
%		\draw[densely dotted]  (#2+.4,#1)--(#2+.4,#1+1) ; \node[V] at (#2+.4,#1+.3){}; \node[V] at  (#2+.4,#1+.7){};}

\def\Tops[#1][#2][#3]{%1=pole locations, 2=top, 3=k 
	\foreach\x in {#1}{
		\draw (#2,\x+.15) -- (#2+.1, \x+.15) (#2, \x-.15) -- (#2+.1, \x-.15) ;
		\draw (#2+.1,\x) arc (0:360:.75mm and 1.5mm);}
	%Nodes 
	\foreach \x in {1,...,#3} {\draw (#2,\x)  to (#2+.05,\x); \node[V] at (#2+.05,\x){};}
	}
\def\Bottoms[#1][#2][#3]{%1=pole locations, 2 = bottom, 3=top 
	\foreach\x in {#1}{
		\draw (#2, \x+.15) -- (#2-.1, \x+.15) (#2, \x-.15) -- (#2-.1, \x-.15) ;
		\draw (#2-.1, \x+.15) arc (90:270:.75mm and 1.5mm);}
	%Nodes 
	\foreach \x in {1,...,#3} {\draw (#2, \x)  to (#2-.05, \x); \node[V] at (#2-.05, \x){};}
	}
\def\Caps[#1][#2,#3][#4]{%1=pole locations, 2 = bottom, 3=top, 4=k 
	\Tops[#1][#3][#4]
	\Bottoms[#1][#2][#4]
	}
\def\Pole[#1][#2,#3]{%1=horizontal location, 2 = bottom, 3=top
	\shade[left color=white,right color=white] (#2,#1+.15) rectangle (#3,#1-.15);
	\draw[over] (#2,#1+.15) to (#3,#1+.15) (#2,#1-.15) to (#3,#1-.15) ;}
\def\Label[#1,#2][#3][#4]{%1,2 = top/bot position, 3=i, 4=label
	\node[right] at (#2+.1,#3) {#4};
	\node[left] at (#1-.1,#3) {#4};		}
\def\Nodes[#1][#2]{
	 \foreach \x in {1,...,#2} {\node[V] at (#1,\x){};	}
	}
\def\PoleCaps[#1][#2,#3]{%1=pole location, 2 = bottom, 3=top, 4=k 
	\foreach\x in {#1}{
		\draw (#2,\x+.15) -- (#2-.1,\x+.15) (#2,\x-.15) -- (#2-.1,\x-.15) ;
		\draw (#2-.1,\x+.15) arc (0:-180:1.5mm and .75mm);}
	\foreach\x in {#1}{
		\draw (#3,\x+.15) -- (#3+.1,\x+.15) (#3,\x-.15) -- (#3+.1,\x-.15) ;
		\draw (#3+.1,\x+.15) arc (0:360:1.5mm and .75mm);}
	}
\def\PoleTwist[#1,#2]{%1 = bottom, 2=top
	\foreach \x/\y in {-1/1L, -.7/1R, 0/2L, .3/2R}{\coordinate(T\y) at (#2,\x); \coordinate(B\y) at (#1,\x);}
	\draw[thin] (B1R) .. controls ++(#2*.5-#1*.5-.1,0) and ++(-#2*.5+#1*.5-.1,0) ..  (T2R)
			(B1L)   .. controls ++(#2*.5-#1*.5+.1,0) and ++(-#2*.5+#1*.5+.1,0) ..    (T2L) ;
	\draw[line width=2pt, white]
			(#1,.15)  .. controls +(#2*.5-#1*.5,0) and +(-#2*.5+#1*.5,0) ..   (#2,-.85) ;
	\draw[thin,over] 
		(B2R) .. controls ++(#2*.5-#1*.5+.1,0) and ++(-#2*.5+#1*.5+.1,0) ..  (T1R) 
			(B2L)  .. controls +(#2*.5-#1*.5-.1,0) and +(-#2*.5+#1*.5-.1,0) ..   (T1L) ;
			}


\def\SymPolesCaps[#1,#2][#3]{%1 = Vertical position, 2=k
	\draw (#1,.3) -- (#1-.1,.3) (#1,.15) -- (#1-.1, .15) ;
	\draw (#1-.1, .3) arc (0:-180:2pt and 1.5pt);
	\draw (#1,#3+.7) -- (#1-.1,#3+.7) (#1,#3+.85) -- (#1-.1,#3+.85) ;
	\draw (#1-.1,#3+.85)  arc (0:-180:2pt and 1.5pt);
	\draw (#2,.3) -- (#2+.1, .3) (#2, .15) -- (#2+.1, .15) ;
	\draw (#2+.1, .3) arc (0:360:2pt and 1.5pt);
	\draw (#2, #3+.7) -- (#2+.1, #3+.7) (#2, #3+.85) -- (#2+.1, #3+.85) ;
	\draw (#2+.1, #3+.85) arc (0:360:2pt and 1.5pt);}


%************Commutative diagrams**********************

\def\mapright#1{\smash{\mathop
        {\longrightarrow}\limits^{#1}}}

\def\mapleftright#1{\smash{\mathop
        {\longleftrightarrow}\limits^{#1}}}


\def\mapsrightto#1{\smash{\mathop
        {\longmapsto}\limits^{#1}}}

\def\mapleft#1{\smash{
   \mathop{\longleftarrow}\limits^{#1}}}

\def\mapdown#1{\Big\downarrow
   \rlap{$\vcenter{\hbox{$\scriptstyle#1$}}$}}

\def\lmapdown#1{{\hbox{$\scriptstyle#1$}}
\llap {$\vcenter{\hbox{\Big\downarrow}}$} }

\def\mapupdown#1{\Big\updownarrow
   \rlap{$\vcenter{\hbox{$\scriptstyle#1$}}$}}

\def\lmapupdown#1{{\hbox{$\scriptstyle#1$}}
\llap {$\vcenter{\hbox{\Big\updownarrow}}$} }

\def\mapup#1{\Big\uparrow
   \rlap{$\vcenter{\hbox{$\scriptstyle#1$}}$}}
\def\mapne#1{\Big\nearrow
   \rlap{$\vcenter{\hbox{$\scriptstyle#1$}}$}}
\def\mapse#1{
%{$\vcenter{
\hbox{$\scriptstyle#1$}
%$}
\rlap{ $\vcenter{\hbox{$\searrow$}}$ }  }
\def\mapnw#1{\Big\nwarrow
   \rlap{$\vcenter{\hbox{$\scriptstyle#1$}}$}}
\def\mapsw#1{
%\Big
\swarrow
   \rlap{$\vcenter{\hbox{$\scriptstyle#1$}}$}}


\newcommand{\posleq}[1]{
	\hspace{0.1cm}
	\begin{tikzpicture}
	\draw (-0.8ex, -0.5ex) -- (0.8ex, -0.5ex);
	\draw (-0.8ex, 0.4ex) -- (0.7ex, -0.2ex);
	\draw (-0.8ex, 0.4ex) -- (0.7ex, 1ex);
	\draw (0.4ex,0.4ex) --(1.1ex, 0.4ex);
	\draw (0.75ex,0.75ex) --(0.75ex, 0.05ex);
	\end{tikzpicture}
	\hspace{0.1cm}
	}
\newcommand{\negleq}[1]{
	\hspace{0.1cm}
	\begin{tikzpicture}
	\draw (-0.8ex, -0.5ex) -- (0.8ex, -0.5ex);
	\draw (-0.8ex, 0.4ex) -- (0.7ex, -0.2ex);
	\draw (-0.8ex, 0.4ex) -- (0.7ex, 1ex);
	\draw (0.4ex,0.4ex) --(1.1ex, 0.4ex);
	\end{tikzpicture}
	\hspace{0.1cm}
	}
	
\newcommand{\zeroleq}[1]{
	\hspace{0.1cm}
	\begin{tikzpicture}
	\draw (-0.8ex, -0.5ex) -- (0.8ex, -0.5ex);
	\draw (-0.8ex, 0.4ex) -- (0.7ex, -0.2ex);
	\draw (-0.8ex, 0.4ex) -- (0.7ex, 1ex);
	\draw  (0.75ex,0.4ex) ellipse (0.2ex and 0.35ex);
	\end{tikzpicture}
	\hspace{0.1cm}
	}
	
\newcommand{\posgeq}[1]{
	\hspace{0.1cm}
	\begin{tikzpicture}
	\draw (-0.8ex, -0.5ex) -- (0.8ex, -0.5ex);
	\draw (0.8ex, 0.4ex) -- (-0.7ex, -0.2ex);
	\draw (0.8ex, 0.4ex) -- (-0.7ex, 1ex);
	\draw (-0.4ex,0.4ex) --(-1.1ex, 0.4ex);
	\draw (-0.75ex,0.75ex) --(-0.75ex, 0.05ex);
	\end{tikzpicture}
	\hspace{0.1cm}
	}
\newcommand{\neggeq}[1]{
	\hspace{0.1cm}
	\begin{tikzpicture}
	\draw (-0.8ex, -0.5ex) -- (0.8ex, -0.5ex);
	\draw (0.8ex, 0.4ex) -- (-0.7ex, -0.2ex);
	\draw (0.8ex, 0.4ex) -- (-0.7ex, 1ex);
	\draw (-0.4ex,0.4ex) --(-1.1ex, 0.4ex);
	\end{tikzpicture}
	\hspace{0.1cm}
	}
	
\newcommand{\zerogeq}[1]{
	\hspace{0.1cm}
	\begin{tikzpicture}
	\draw (-0.8ex, -0.5ex) -- (0.8ex, -0.5ex);
	\draw (0.8ex, 0.4ex) -- (-0.7ex, -0.2ex);
	\draw (0.8ex, 0.4ex) -- (-0.7ex, 1ex);
	\draw  (-0.75ex,0.4ex) ellipse (0.2ex and 0.35ex);
	\end{tikzpicture}
	\hspace{0.1cm}
	}

\newcommand{\posl}[1]{
	\hspace{0.1cm}
	\begin{tikzpicture}
	\draw (-0.8ex, 0.4ex) -- (0.7ex, -0.2ex);
	\draw (-0.8ex, 0.4ex) -- (0.7ex, 1ex);
	\draw (0.4ex,0.4ex) --(1.1ex, 0.4ex);
	\draw (0.75ex,0.75ex) --(0.75ex, 0.05ex);
	\end{tikzpicture}
	\hspace{0.1cm}
	}
\newcommand{\negl}[1]{
	\hspace{0.1cm}
	\begin{tikzpicture}
	\draw (-0.8ex, 0.4ex) -- (0.7ex, -0.2ex);
	\draw (-0.8ex, 0.4ex) -- (0.7ex, 1ex);
	\draw (0.4ex,0.4ex) --(1.1ex, 0.4ex);
	\end{tikzpicture}
	\hspace{0.1cm}
	}
	
\newcommand{\zerol}[1]{
	\hspace{0.1cm}
	\begin{tikzpicture}
	\draw (-0.8ex, 0.4ex) -- (0.7ex, -0.2ex);
	\draw (-0.8ex, 0.4ex) -- (0.7ex, 1ex);
	\draw  (0.75ex,0.4ex) ellipse (0.2ex and 0.35ex);
	\end{tikzpicture}
	\hspace{0.1cm}
	}
	
\newcommand{\posg}[1]{
	\hspace{0.1cm}
	\begin{tikzpicture}
	\draw (0.8ex, 0.4ex) -- (-0.7ex, 1ex);
	\draw (0.8ex, 0.4ex) -- (-0.7ex, -0.2ex);
	\draw (-0.4ex,0.4ex) --(-1.1ex, 0.4ex);
	\draw (-0.75ex,0.75ex) --(-0.75ex, 0.05ex);
	\end{tikzpicture}
	\hspace{0.1cm}
	}
\newcommand{\negg}[1]{
	\hspace{0.1cm}
	\begin{tikzpicture}
	\draw (0.8ex, 0.4ex) -- (-0.7ex, -0.2ex);
	\draw (0.8ex, 0.4ex) -- (-0.7ex, 1ex);
	\draw (-0.4ex,0.4ex) --(-1.1ex, 0.4ex);
	\end{tikzpicture}
	\hspace{0.1cm}
	}
	
\newcommand{\zerog}[1]{
	\hspace{0.1cm}
	\begin{tikzpicture}
	\draw (0.8ex, 0.4ex) -- (-0.7ex, -0.2ex);
	\draw (0.8ex, 0.4ex) -- (-0.7ex, 1ex);
	\draw  (-0.75ex,0.4ex) ellipse (0.2ex and 0.35ex);
	\end{tikzpicture}
	\hspace{0.1cm}
	}
\usepackage{soul,color, xcolor} 
\usepackage[authoryear]{natbib}
\usepackage{setspace}
\usepackage{bbm}

\pagestyle{fancy}
\fancyhf{} 
\fancyfoot[C]{\thepage} 
\fancyhead[R]{\mbox{\leftmark}} 

\onehalfspacing
\bibliographystyle{alpha}

\title{Report about group project part 2}
\author{Group six: Zihan Lin,Kevin Tran,Brice Arrigo \& Jude Hine}
\date{\today}
\usepackage{graphicx} 
\usepackage{float} 
\usepackage{subfigure}
\usepackage{float}
\usepackage{parskip}

\begin{document}

\begin{titlepage}
\vspace*{\stretch{1}}
\begin{center}
	{\Huge\bfseries AMSI Summer School -- 2026}\\[3.5ex]
  	{\huge\bfseries Computational and Combinatorial Algebraic Topology: Group Project -- Report}                  \\[6.5ex]
  	{\large\bfseries Group 6: Zihan Lin, Kevin Tran, Brice Arrigo \& Jude Hine}           \\
  	\today
\end{center}
\vspace{\stretch{2}}
\end{titlepage}
\newpage
\begin{abstract}
% This article is a report of part 2 of group project of Computational and combinatorial algebraic topology.

% We use python to create a software tool to help me research, which is the attachment "project 2 code of group six".

% This is a report on the computational project for Computational and Combinatorial Algebraic Topology. We employ the GUDHI library in Python to use persistent homology to classify geometric configurations of planar point clouds. We then show how persistence barcodes can be used to detect outliers from noisy data. 
This report details the computational project for Computational and Combinatorial Algebraic Topology at AMSI Summer School 2026. Using the GUDHI library in Python, we apply persistent homology to classify different geometric configurations of planar point clouds. In addition, we demonstrate how persistence barcodes can be used to identify and remove outliers in noisy data, recovering the underlying topological structure.
\end{abstract}

~\\
\tableofcontents


\newpage
\section{Generate the dataset}\label{sec-i}

The code for generating circles is adapted from the code presented in the week three tutorial. 

The following function generates a point cloud containing 200 points, sampling from two disjoint concentric noisy circles.
\begin{lstlisting}
def generate_two_disjoint_concentric_noisy_circle(n=200, radius1=1,radius2=2, center=(0, 0), noise_std=0.1):
	theta = np.random.uniform(0.0, 2.0 * np.pi, size=100)
	ux = np.cos(theta)
	uy = np.sin(theta)
	cx, cy = center
	base = np.column_stack([cx + radius1 * ux, cy + radius1 * uy])
	noise = np.random.rand(100,2)
	points1 = base + noise_std*radius1*noise
	theta = np.random.uniform(0.0, 2.0 * np.pi, size=100)
	ux = np.cos(theta)
	uy = np.sin(theta)
	cx, cy = center
	base = np.column_stack([cx + radius2 * ux, cy + radius2 * uy])
	noise = np.random.rand(100, 2)
	points2 = base + noise_std * radius1 * noise
	points = np.vstack([points1, points2])
	return points
\end{lstlisting}

The code below generates a point cloud containing 200 points, sampling from two disjoint noisy circles that are not nested.
\begin{lstlisting}
def generate_two_disjoint_noisy_circle(n=200,radius=1, center1=(0,0), center2=(3,0), noise_std=0.1):
    theta = np.random.uniform(0.0, 2.0 * np.pi, size=100)
    ux = np.cos(theta)
    uy = np.sin(theta)
    cx1, cy1 = center1
    base = np.column_stack([cx1 + radius * ux, cy1 + radius * uy])
    noise = np.random.rand(100, 2)
    points1 = base + noise_std * radius * noise
    theta = np.random.uniform(0.0, 2.0 * np.pi, size=100)
    ux = np.cos(theta)
    uy = np.sin(theta)
    cx2, cy2 = center2
    base = np.column_stack([cx2 + radius * ux, cy2 + radius * uy])
    noise = np.random.rand(100, 2)
    points2 = base + noise_std * radius * noise
    points = np.vstack([points1, points2])
    return points
\end{lstlisting}

The code below generates a point cloud containing 200 points, sampling from two adjacent noisy circles.
\begin{lstlisting}
def generate_two_adjacent_noisy_circle(n=200,radius=1, center1=(0,0), center2=(2,0), noise_std=0.1):
    theta = np.random.uniform(0.0, 2.0 * np.pi, size=100)
    ux = np.cos(theta)
    uy = np.sin(theta)
    cx1, cy1 = center1
    base = np.column_stack([cx1 + radius * ux, cy1 + radius * uy])
    noise = np.random.rand(100, 2)
    points1 = base + noise_std * radius * noise
    theta = np.random.uniform(0.0, 2.0 * np.pi, size=100)
    ux = np.cos(theta)
    uy = np.sin(theta)
    cx2, cy2 = center2
    base = np.column_stack([cx2 + radius * ux, cy2 + radius * uy])
    noise = np.random.rand(100, 2)
    points2 = base + noise_std * radius * noise
    points = np.vstack([points1, points2])
    return points
\end{lstlisting}

We store the three classes of point clouds in a numpy array, \texttt{dataset}.
\begin{lstlisting}
dataset = []
for i in range(0,3):
    for j in range(0,100):
        if i == 0:
            pc = generate_two_disjoint_concentric_noisy_circle()
        elif i == 1:
            pc = generate_two_disjoint_noisy_circle()
        else:
            pc = generate_two_adjacent_noisy_circle()
        dataset.append(pc)
dataset=np.array(dataset)
\end{lstlisting}
  
\section{Cluster the dataset}\label{sec-ii}

We first analyse the persistence barcodes qualitatively in order to identify distinguishing features between the different classes. These features are then quantified via suitable interval counting heuristics on the persistence diagrams, which are used to cluster our dataset.

% We begin by making qualitative observations on the level of the persistence barcode for each class. From these observations we determine statistics which we can associate to each persistence diagram. The statistics are then used to cluster our dataset.  

% Using the persistence barcode, we detect differences in persistent homology for our data set, allowing us to classify the three classes. 

\cref{fig:conc_circ} is the persistence barcode of the point cloud obtained from sampling two disjoint concentric noisy circles.
\begin{figure}[h!] 
\centering 
\includegraphics[width=1\textwidth]{q_ii_nested_barcode.png} 
\caption{persistence barcode of sample from two concentric circles.}
\label{fig:conc_circ}
\end{figure}
\newpage
\cref{fig:adj_circ} is a persistence barcode of the point cloud obtained from sampling two adjacent noisy circles.
\begin{figure}[h!] 
\centering 
\includegraphics[scale=1]{3} 
\caption{persistence barcode of sample from two adjacent circles.}
\label{fig:adj_circ}
\end{figure}

\newpage
\cref{fig:disj_circ} is a persistence barcode of the point cloud obtained from sampling two disjoint noisy circles that are not nested.

\begin{figure}[h!] 
\centering 
\includegraphics[scale=1]{2} 
\caption{persistence barcode of sample from two disjoint circles.}
\label{fig:disj_circ}
\end{figure}


Examining dimension \(0\) in \cref{fig:conc_circ}, persistence intervals with significantly longer lifetimes than those observed in the other two classes. 
% This indicates that the sum of the lifetime of each barcode in dimension \(0\) would be a suitable statistic for capturing this behaviour. 
 
Looking at dimension \(1\) of \cref{fig:conc_circ}, we see the persistence barcodes are noticeably different from the other two classes, \cref{fig:adj_circ,fig:disj_circ}, whose dimension \(1\) barcodes consists of only two intervals with a relatively long lifespan. 

We see that \cref{fig:adj_circ} has only one long barcode in dimension \(0\), since the adjacent circles only have one connected component. But the other two classes have more the one long dimension \(0\) barcode.

So, in summary, we will check for the following properties: 
\begin{enumerate}
	\item if dimension \(0\) has two very long intervals, then the point cloud is comes from either nested or disjoint circles;
	\item if dimension \(1\) has many intervals, then the point cloud comes from two nested circles; and
	\item if it is neither of the two, then it must be two adjacent circles. 
\end{enumerate} 

We first present a psuedocode version of the algorithm, followed by its implementation in python. 

\begin{lstlisting}
	Input: data - nested array of point clouds
	Output: classification - list with elements in [0,1,2], corresponding to nested, disjoint and adjacent circles respective.
	
	Let PH = []
	Let classification = []
	for each point cloud in data
		Compute Rips filtration
		Calculate persistence
		Let H\_0, H\_1 = persistence intervals for dimensions 0 and 1
		Append [H\_0,H\_1] to PH  

	for each persistence interval in PH
		if the largest finite death time H\_0 >= 0.8 and the second largest finite death time H\_0 >= 0.2
			if the number of intervals in H\_1 >= 12
				the point cloud comes from two nested circles
				append 0 to classification
			else:
				the point cloud comes from two disjoint circles
				append 1 to classification
		if the largest finite death time H\_0 <= 0.5
			the point cloud comes from two adjacent circles
			append 2 to classification
	return classification	 
\end{lstlisting}
Where the constants were chosen empirically through trial and error. This was done by examining the largest persistence lifetimes for each barcode.   


This is practically already the python algorithm, which is as follows 

\begin{lstlisting}
def classify_my_data(data):
    PH = []
    classification = []
    for points in data:
        rips = gudhi.RipsComplex(points=points)
        st = rips.create_simplex_tree(max_dimension=2)
        st.persistence()
        H0 = st.persistence_intervals_in_dimension(0)
        H1 = st.persistence_intervals_in_dimension(1)
        PH.append([H0,H1])
        
    for persistence_interval_pair in PH:
        persistence_0, persistence_1 = persistence_interval_pair[0], persistence_interval_pair[1]
        persistence_0, persistence_1 = persistence_0[::-1], persistence_1[::-1]
        if persistence_0[1][1]>=0.8 and persistence_0[2][1] >= 0.2:
            if len(persistence_1) >= 12:
                classification.append(0)
            else:
                classification.append(1)
        if persistence_0[1][1]<=0.5: 
            classification.append(2)
    return classification

result = classify_my_data(dataset)
np.unique(result,return_counts=True)
\end{lstlisting}
The last line returns the frequency of each element in the list. Which running on our generated dataset returns 
\begin{lstlisting}
	(array([0, 1, 2]), array([100, 100, 100], dtype=int64))
\end{lstlisting} 

So we have successfully used persistent homology to come up with a heuristics approach to classifying our point clouds. While there are clear opportunities to refine these heuristics and improve robustness, this method proves to be highly effective for the dataset under consideration. We will see in \cref{sec-iii} how this approach breaks down with more noise.

% This suggests the mean of the lifetime of each barcode in dimension \(1\) would be a suitable statistic.

% Additionally, the number of homology classes and the maximum lifetime for each dimension appear to be useful for distinguishing between the three types of circle configurations. 

% \newpage
% So in summary, we will collect the following statistics to help us cluster the dataset:
% \begin{enumerate}
% 	\item number of homology classes; \label{item:list:homology_count}
% 	\item mean;\label{item:list:mean}
% 	\item maximum; and \label{item:list:max}
% 	\item sum.\label{item:list:sum}
% \end{enumerate}


% The persistence barcodes show that persistent homology could help us cluster the dataset.
% For example, the statistics for concentric nested, disjoint and adjacent circles look as follows:
% \begin{center}
% \begin{varwidth}{\linewidth}
% \begin{verbatim}
% [array([1.99000000e+02, 1.04442946e-01, 8.99211686e-01, 2.07841463e+01,
%         1.90000000e+01, 1.19393328e-01, 1.26936331e+00, 2.26847324e+00]),
%  array([1.99000000e+02, 7.94405605e-02, 9.80034573e-01, 1.58086715e+01,
%         6.00000000e+00, 4.68835861e-01, 1.41398329e+00, 2.81301517e+00]),
%  array([1.99000000e+02, 7.23665151e-02, 2.34194535e-01, 1.44009365e+01,
%         1.40000000e+01, 2.09510741e-01, 1.44064551e+00, 2.93315037e+00])]
% \end{verbatim}
% \end{varwidth}
% \end{center} 
% Where each array consists of statistics \labelcref{item:list:homology_count,item:list:mean,item:list:max,item:list:sum} for dimension \(0\) followed by dimension \(1\), respectively. 

% Using these arrays, we can either manually come up with rules for how the statistics determine which class a persistence diagram belongs to or we can use KMeans as a method for clustering. We opted to use KMeans as it is clean and quite efficient. 

% Using these arrays, one could manually construct classification rules based on the observed statistics of the persistence diagrams. Instead, we employ k-means clustering, which offers a straightforward and computationally efficient unsupervised approach. This allows the data to be grouped according to similarities in their persistence-based features, providing an effective means of distinguishing the three types of circle pair configurations without relying on ad hoc thresholds.

% For this report, we chose to use KMeans, as we wanted to learn about it. 
% We now present the Python implementation of our algorithm.    
% \begin{lstlisting}
% PH = []
% for i in range(0, 300):
%     rips_complex = gudhi.RipsComplex(points=dataset[i])
%     st = rips_complex.create_simplex_tree(max_dimension=2)
%     a = st.persistence()
%     PH.append(ph(a))

% PH = np.array(PH)
% kmeans = KMeans(n_clusters=3, random_state=0)
% clusters = kmeans.fit_predict(PH)
% clusters
% \end{lstlisting}
% \newpage

% Where \texttt{ph()} is an auxiliary function used to compute the statistics on the level of persistent homology.  

% \begin{lstlisting}
% def split_dgms(persistence):
%     dgm0 = []
%     dgm1 = []
%     for dim, (b, d) in persistence:
%         if d == float("inf"):
%             continue
%         if dim == 0:
%             dgm0.append([b, d])
%         elif dim == 1:
%             dgm1.append([b, d])
%     return np.array(dgm0), np.array(dgm1)

% def diagram_stats(dgm):
%     if len(dgm) == 0:
%         return np.zeros(4)
%     pers = dgm[:,1] - dgm[:,0]
%     return np.array([len(pers), pers.mean(), pers.max(), pers.sum()])

% def ph(persistence):
%     dgm0, dgm1 = split_dgms(persistence)
%     v0 = diagram_stats(dgm0)
%     v1 = diagram_stats(dgm1)
%     return np.concatenate([v0, v1])
% \end{lstlisting}

% This code will output a vector consisting of one hundred \(0\)s, one hundred \(1\)s and one hundred \(2\)s.

% Since our dataset consists of 100 two disjoint concentric noisy circles firstly, then generated 100 two disjoint noisy circles that are not nested before generating 100 two adjacent noisy circles. So this code clusters the dataset successfully.
% This code outputs a vector consisting of one hundred entries equal to \(0\), one hundred entries equal to \(1\), and one hundred entries equal to \(2\). Since the dataset was constructed by first generating \(100\) samples of two nested noisy circles, followed by \(100\) samples of two disjoint noisy circles, and finally \(100\) samples of two adjacent noisy circles, this output indicates that the algorithm has correctly clustered the dataset into the three intended classes. The clustering therefore successfully recovers the underlying structure of the data generation process.

\section{Noise Influence}\label{sec-iii}

In \cref{sec-i} when generating our dataset, we fixed \texttt{noise\_std = 0.1}. We observed setting \texttt{noise\_std = 0.3} fail to classify most of the data. For example, 
\begin{lstlisting}
	(array([0, 1, 2]), array([ 45,   1, 100], dtype=int64))
\end{lstlisting}
indicates that while the algorithm continues to correctly identify the adjacent circle configurations, the increased variance in the point clouds makes it difficult to reliably distinguish between nested and disjoint circles. In particular, some of the detected classes may correspond to false positives.

Increasing further to \texttt{noise\_std = 0.5}, causes the classification procedure to fail almost entirely, as the persistence-based heuristics no longer provide sufficient separation between the different configurations. 
For example, 
\begin{lstlisting}
	(array([0, 2]), array([  1, 116], dtype=int64))
\end{lstlisting}
shows that the algorithm is no longer able to detect the disjoint circle configuration, as the corresponding cluster is completely absent.

This problem makes sense within the theoretical framework of persistent homology. This is because the distance between two disjoint concentric noisy circles or two disjoint noisy circles that are not nested are both \(1\). So when noise is over 0.5, the differences in persistence homology among them will be distorted. For example, two connected components may be considered one connected component now, which will influence \(\mathrm{PH}_0\).

Further empirical evidence showing the influence of noise on the accuracy of our algorithm is the following noisy point cloud with and its associated barcode.  
\begin{figure}[!htb]
\begin{minipage}{0.5\textwidth}
	\centering
	\includegraphics[width=\linewidth, height=\linewidth]{too_noisy.png}
	\caption{Point cloud sampled from two concentric circles with \texttt{noise\_std = 0.5}.}
	\label{fig:too_noisy}
\end{minipage}
\begin{minipage}{0.5\textwidth}
	\centering
	\includegraphics[width=\linewidth, height=\linewidth]{q_iii_noisy_barcode.png}
	\caption{Barcode of Rips complex of point cloud.}
	\label{fig:noisy_barcode}
\end{minipage}
\end{figure}

We can see \cref{fig:noisy_barcode} has a noticeably different structure in dimension \(1\), to that of it's less noisy counterpart \cref{fig:conc_circ}. Visually showing how increased sampling noise leads to less accurate classification based on persistent homology.  

% If we still increase the noise, the persistent homology of three classes will appear to be the same for this code. 
\newpage
\section{Outlier points}\label{sec-iv}
The code below generates a point cloud, \(P_1\), consisting of \(200\) points, sampling a noisy circle and creates a second point cloud \(P_2\) by adding \(20\) outlier points to \(P_1\). We then display the persistence diagrams of \(P_1\) and \(P_2\)
\begin{lstlisting}
def generate_random_points(n=20, xlim=(-1.0, 1.0), ylim=(-1.0, 1.0)):
	x = np.random.uniform(xlim[0],xlim[1],size=n)
	y = np.random.uniform(xlim[0],xlim[1],size=n)
	return np.column_stack([x, y])

def generate_noisy_circle(n=200, radius=1, center=(0, 0), noise_std=0.1):
	theta = np.random.uniform(0.0, 2.0 * np.pi, size=n)
	ux = np.cos(theta)
	uy = np.sin(theta)
	cx, cy = center
	base = np.column_stack([cx + radius * ux, cy + radius * uy])
	noise = np.random.rand(n,2)
	points = base + noise_std*radius*noise
	return points

P1 = generate_noisy_circle()
P2 = np.vstack((P1, generate_random_points()))
rips_complex = gudhi.RipsComplex(points=P1)
st = rips_complex.create_simplex_tree(max_dimension=2)
a = st.persistence()
gudhi.plot_persistence_barcode(a, legend=True)
plt.show()
rips_complex = gudhi.RipsComplex(points=P2)
st = rips_complex.create_simplex_tree(max_dimension=2)
a = st.persistence()
gudhi.plot_persistence_barcode(a, legend=True)
plt.show()
\end{lstlisting}

\(P_1\) is a noisy circle, with only one connected component and one cycle. So only \(1\) dimension of $\mathrm{PH}_0$ and $\mathrm{PH}_1$ will live for a long time. In contrast, the outlier points in \(P_2\) will lead to more long living classes in persistent homology since some of them may be far from the circle, forming new connected components and edges when the filtration index is small.
\newpage
We see this information in the persistence barcodes of \(P_1\) and \(P_2\) in \cref{fig:P1-barcode,fig:P2-barcode} respectively. 

\begin{figure}[h!]
\begin{minipage}{0.5\textwidth}
	\centering 
	\includegraphics[width=\linewidth, height=\linewidth]{q_iv_P1.png} 
	\caption{Persistence diagram of \(P_1\).}
	\label{fig:P1-barcode}
\end{minipage}
\begin{minipage}{0.5\textwidth} 
	\centering 
	\includegraphics[width=\linewidth, height=\linewidth]{q_iv_P2.png} 
	\caption{Persistence diagram of \(P_2\).}
	\label{fig:P2-barcode}
\end{minipage}
\end{figure}

Notice, the influence of outlier points is visually clear from \cref{fig:P2-barcode}. With these observations, we are ready to recover a persistence diagram similar to \(P_1\). 

\newpage
\section{Removing outliers}\label{sec-v}

As discussed in \cref{sec-iv}, the points further away from the circle have greater influence over the persistent homology of \(P_2\). So to recover a persistence diagram similar to that of \(P_1\), we will remove points which are far away from other points.

Let the \emph{neighbourhood} of a point $p$ be the area where the distance between $p$ and any point in this area is less than \(0.3\). A point \(p\) is an \emph{outlier point} if the neighbourhood of \(p\) contains fewer than \(5\) points, we will remove any outlier point from the point cloud. Note, however, the points we remove may not be the \(20\) outlier points we added before. 

The following is an implementation of this outlier removal algorithm. 

\begin{lstlisting}
D = cdist(P2, P2)
neighbors = (D < 0.3).sum(axis=1)
clean = neighbors > 5
P2_clean = P2[clean]
rips_complex = gudhi.RipsComplex(points=P2_clean)
st = rips_complex.create_simplex_tree(max_dimension=2)
a = st.persistence()
gudhi.plot_persistence_barcode(a, legend=True)
plt.show()
\end{lstlisting}
Where \texttt{P2\_clean} is the recovered point could, with persistence barcode \cref{fig:clean-P2}

\begin{figure}[h!] 
\centering 
\includegraphics[scale=0.7]{q_v_P2_clean.png} 
\caption{Persistence barcode for cleaned \(P_2\).}
\label{fig:clean-P2}
\end{figure}
Which is similar to the persistence diagram for \(P_1\). This demonstrates that removing points which have only a small number of neighbours within a suitably chosen open neighbourhood is effective at removing outliers and recovering the underlying topological structure of the data.


\section{Conclusion}
In conclusion, we have demonstrated the utility of persistent homology as a tool for data analysis. However, as with all tools, it has its limitations. The performance of persistence-based methods is sensitive to noise and sampling density, which can obscure meaningful topological features. Moreover, the computational cost of Vietoris-Rips filtrations and the interpretation of persistence diagrams must be taken into account. Nevertheless, persistent homology remains a powerful approach for extracting qualitative structural information from complex data.
\end{document}
